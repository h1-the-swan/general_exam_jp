\section{Applications of community
detection}\label{applications-of-community-detection}

The extensive work behind developing and validating methods for
community detection are presumably meant to work toward a goal of using
community detection for some concrete applications. In this area, the
field shows its (young) age---it is somewhat difficult to find published
examples where the methods have been applied to solve a specific problem
or gain significant new insight into a system. Below, I discuss some of
the examples that do exist in the research literature, in the fields of
\protect\hyperlink{social-network-analysis}{social network analysis},
\protect\hyperlink{networks-of-scholarship-and-scholarly-communication}{networks
of scholarship and scholarly communication},
\protect\hyperlink{biological-networks}{biological networks}, and
\protect\hyperlink{other-research}{others}. Besides published research,
I speculate on the (mostly unpublished) applications of community
detection outside of academia, and present an example of using community
detection in the context of a small data science project to address a
question of interest.

\hypertarget{social-network-analysis}{\subsection{Social network
analysis}\label{social-network-analysis}}

\autocite{traud_comparing_2011} Comparing Community Structure to
Characteristics in Online Collegiate Social Networks

\autocite{weng_virality_2013} Applied Infomap to help predict virality
of memes.

\hypertarget{networks-of-scholarship-and-scholarly-communication}{\subsection{Networks
of scholarship and scholarly
communication}\label{networks-of-scholarship-and-scholarly-communication}}

\hypertarget{biological-networks}{\subsection{Biological
networks}\label{biological-networks}}

\autocites{bacik_flow-based_2016}{holme_subnetwork_2003}

\hypertarget{other-research}{\subsection{Other examples in the research
literature}\label{other-research}}

\TODO{Legislative networks: see porter paper}

\begin{itemize}
\tightlist
\item
  \autocite{lupu_trading_2013}: States that trade with each other are
  known to have less conflict with each other. This paper extends this
  to trade communities, so that states that don't trade with each other
  that much, but are indirectly linked by trade, also have less conflict
  with each other. They use community detection (via modularity
  maximization) to test this theory.
\end{itemize}

\TODO{see fortunato paper. see porter paper.}

\subsection{Non-research applications}\label{non-research-applications}

\TODO{terrorism? crime? marketing? maybe not reflected in the research. Community detection to identify fraud events in telecommunication networks.}
