\section{Applications of community
detection}\label{applications-of-community-detection}

The extensive work behind developing and validating methods for
community detection are presumably meant to work toward a goal of using
community detection for some concrete applications. In this area, the
field shows its (young) age---it is somewhat difficult to find published
examples where the methods have been applied to solve a specific problem
or gain significant new insight into a system. Below, I discuss some of
the examples that do exist in the research literature, in the fields of
\protect\hyperlink{social-network-analysis}{social network analysis},
\protect\hyperlink{networks-of-scholarship}{networks of scholarship},
\protect\hyperlink{biological-networks}{biological networks}, and
\protect\hyperlink{other-research}{others}. Besides published research,
I speculate on the (mostly unpublished) applications of community
detection outside of academia, and present an example of using community
detection in the context of a small data science project to address a
question of interest.

\hypertarget{social-network-analysis}{\subsection{Social network
analysis}\label{social-network-analysis}}

\autocite{traud_comparing_2011} Comparing Community Structure to
Characteristics in Online Collegiate Social Networks

\autocite{weng_virality_2013} Applied Infomap to help predict virality
of memes.

\hypertarget{networks-of-scholarship}{\subsection{Networks of
scholarship}\label{networks-of-scholarship}}

The collective human endeavor of knowledge generation and organization
can be represented as a directed network of scholarly publications with
citations between them. The citation links between publications can been
viewed as a proxy for influence or information flow. De Solla Price
recognized the potential of this representation in the mid 20th century
\autocite{de_solla_price_networks_1965}, and over the years much work
has been done in this meta-scientific research area, which has been
given terms such as ``bibliometrics,'' ``scientometrics,'' and ``science
of science''.
\TODO{Communities represent fields of study. Dynamic methods are good for this. They analyzed JSTOR. Can identify related literature and possibly unknown connections. Recommendation paper. Cultural holes? Mapping change--neuroscience?}

\hypertarget{biological-networks}{\subsection{Biological
networks}\label{biological-networks}}

Many biological systems can be thought of as structured interactions
between functional elements, often with (possibly hierarchical) modular
structure. It is natural to think of these systems as networks, and to
see promise in the prospect of identifying communities in this network
representation.

Protein-protein interaction networks are constructed from data collected
in experiments that identify molecular interactions between proteins.
Community detection on these networks, for example using the
Girvan-Newman edge betweenness method, has been shown to be effective in
identifying what are known as ``functional modules'' in these
networks---groups of proteins that interact in the service of a
particular cellular process. Clusters found in these networks correspond
to existing annotations, suggesting promise for the automated analysis
of experiments. These methods were also found to be robust against false
positive interactions, which is important considering that experimental
results can contain considerable noise \autocite{dunn_use_2005}. Chen
and Yuan \autocite{chen_detecting_2006} integrated multiple protein
interaction datasets containing hundreds of microarray expression
profiles for \emph{Saccharomyces cerevisiae} (brewer's yeast) to form a
weighted graph, in which the weights correspond to dissimilarity between
genes' expression profiles. By classifying the genes into functional
modules using a modified version of the Girvan-Newman method, they were
able to predict the function of the as-yet not annotated yeast gene
\emph{YLR419w} to be chromosome segregation.

Another biological network that has been studied is the directed network
of neuronal connections---the ``connectome''. Dynamically-minded
community detection methods (see subsection
``\protect\hyperlink{the-dynamical-perspective}{The dynamical
perspective}'' in section
``\protect\hyperlink{community-detection-methods}{Community detection
methods}'' above) are especially relevant for these networks as the
connectome represents a system of information flow, which is what these
methods model. Bacik et al. \autocite{bacik_flow-based_2016} grouped the
neurons of \emph{C. elegans} using flow-based methods---these detected
communities showed good agreement with previous understanding of
functional neuronal groups. They were then able to perform \emph{in
silico} ablations of neurons---computer simulations in which they
removed nodes and looked at the resulting disruption on community
structure. By doing this, they identified neurons important to the
network flow, and presumably important to the neuronal function of the
organism. These included neurons known to be important, as well as
previously uninvestigated neurons that can be candidates for future
study.

\TODO{Metabolic networks? Maybe can skip this.}

\hypertarget{other-research}{\subsection{Other examples in the research
literature}\label{other-research}}

\TODO{Legislative networks: see porter paper}

\begin{itemize}
\tightlist
\item
  \autocite{lupu_trading_2013}: States that trade with each other are
  known to have less conflict with each other. This paper extends this
  to trade communities, so that states that don't trade with each other
  that much, but are indirectly linked by trade, also have less conflict
  with each other. They use community detection (via modularity
  maximization) to test this theory.
\end{itemize}

\TODO{see fortunato paper. see porter paper.}

\subsection{Non-research applications}\label{non-research-applications}

\TODO{terrorism? crime? marketing? maybe not reflected in the research. Community detection to identify fraud events in telecommunication networks.}
