\hypertarget{visualization}{\section{Visualizing community structure in
networks}\label{visualization}}

\protect\hyperlink{visualization}{}

A recent paper in the EuroVis conference by Vehlow et al.
\autocite{vehlow_state_2015} gives a thorough overview of the state of
the art in visualizing group structure in networks. In addition to
giving a literature survey as well as a taxonomy for these
visualizations, they provide a detailed curated bibliography at
\url{http://groups-in-graphs.corinna-vehlow.com/}. There one finds a
visualization tool for exploring the surveyed literature on this topic,
and also the full tagged bibliography available for download as a BibTeX
file.

One question was somewhat difficult to answer given the format this
bibliography was presented: What are the labs doing work in this area?
While the bibliography presented important papers related to the topic,
they were not organized by lab. In order to address this question, I
perform a community analysis and visualization on the bibliography data
provided. The visualization is available \ldots{} I also use this
endeavor as a first-hand illustration of the utility and challenges
around detecting and visualizing communities in network data.

Starting with data on papers and seeking insight into the organization
of research in different labs seemed like an appropriate situation to
apply community detection methods. I first constructed a co-authorship
network in which nodes represent authors, and weighted links represent
the number of papers in which a pair appear as authors. In addition, I
used the Microsoft Academic API \TODO{cite} to attempt to assign an
affiliation to each author by querying Microsoft Academic Graph for the
author and retrieving the most prevalent affiliation for that author
among the papers returned. I then used
\protect\hyperlink{the-dynamical-perspective}{Infomap} to find a
hierarchical clustering of the nodes. To visualize the results, I
include only the connected components with at least 10 nodes. Each node
represents a single author. The sizes of the nodes correspond to the
amount of flow---the relative importance of an author in this
co-authorship network. The color of each node is assigned based on the
top-level cluster assignment of that author. Clicking on any node of a
connected component will shift the focus to that component, and reassign
colors, this time based on the bottom-level cluster assignment. I also
provide a search box which makes it easier to find specific authors or
affiliations.

\TODO{note that there might be errors due to using names as unique IDs}

\TODO{Results of exploration. Talk about labs}

\TODO{Back to vis. Describe taxonomy}

\TODO{Briefly talk about tasks and evaluation}

\TODO{Discuss open problems}

\begin{itemize}
\tightlist
\item
  Labs:

  \begin{itemize}
  \tightlist
  \item
    Tong Ji Intelligent Big Data Visualization Lab
    (\url{https://idvxlab.github.io/}) headed by Nan Cao
  \item
    MArVL: Monash Adaptive Visualisation Lab
    (\url{http://marvl.infotech.monash.edu/members/}) headed by Kim
    Marriott (includes Tim Dwyer)
  \item
    University of Stuttgart Visualisation Research Centre
    (\url{http://www.visus.uni-stuttgart.de/en/institute.html}) includes
    Prof Daniel Weiskopf, Research associate Dr.~Fabien Beck, and former
    PhD student Corinna Vehlow (these are the authors of the STAR paper)
  \item
    UC Davis Center for Visualization headed by Kwan-Liu Ma. there are
    problems with their website
  \item
    University of Arizona Graph and Map Algorithm (GAMA) Lab headed by
    Stephen Kobourov
  \item
    Johannes Kepler University Linz -- Institute of Computer Graphics --
    Deputy head of the institute Marc Streit. off on its own connected
    component.
  \item
    David Auber at LaBRI works a lot with Daniel Archambault and Tamara
    Munzner at UBC
  \end{itemize}
\end{itemize}

\TODO{Appendix describing methods?}
