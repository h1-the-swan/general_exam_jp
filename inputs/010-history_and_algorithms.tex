\section{History of community
detection}\label{history-of-community-detection}

Guide in Fortunato introduction

\begin{itemize}
\tightlist
\item
  Earliest examples: fortunato p.4
\item
  Social sciences: social ties
\item
  computer science: parallel computing. graph partitioning (since 1970s)
\item
  Girvan and Newman paper 2002. Physicists enter the game.
\end{itemize}

Across many disciplines of science, it is common to encounter data that
can best be represented as a network, with entities linked to each other
through association, flow, or some form of connection. These entities
are represented by \emph{nodes} or \emph{vertices} connected to each
other with \emph{links} or \emph{edges}. This overall representation is
known as a \emph{network} or \emph{graph}. The idea of community
detection as a research field comes from a basic intuition that there
exist in these networks groups of nodes that are structurally more
related to each other than they are to members of other groups. Within
the community comprising the new, interdisciplinary field of
\emph{network science}, the concept of \emph{community} has been
somewhat more formalized as a group of nodes (a \emph{subgraph}) with a
high concentration of edges connecting vertices within the group, and a
low concentration of edges with nodes outside the group
\autocite{fortunato_community_2010}.

The earliest analyses of communities were made by social scientists in
the first half of the twentieth century. \todo{examples}. Then, starting
in the early 1970s, computer scientists started in\ldots{}graph
partitioning for parallel computing and circuit layout.

Then came Girvan and Newman in 2002, ushering in the modern age of
community detection \autocite{lancichinetti_community_2009}.
