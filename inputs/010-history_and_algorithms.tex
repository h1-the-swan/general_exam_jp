\section{History of community
detection}\label{history-of-community-detection}

Across many disciplines of science, it is common to encounter data that
can best be represented as a network, with entities linked to each other
in some meaningful way, such as through association or flow. These
entities are represented by \emph{nodes} or \emph{vertices} connected to
each other with \emph{links} or \emph{edges}. This overall
representation is known as a \emph{network} or \emph{graph}.\footnote{The
  term \emph{graph} generally refers to a mathematical representation of
  data, while \emph{network} usually has additional connotations related
  to the meaning and context associated with the data
  \autocite{porter_communities_2009}; however, as is the case with many
  terms in this area, the distinction is not always made and the two are
  often used interchangeably.} The idea of community detection as a
research topic comes from a basic intuition that there exist in these
networks groups of nodes that are structurally more related to each
other than they are to members of other groups.

The field of \emph{network science} has emerged recently to study this
and related topics. This field is highly interdisciplinary, comprising
physicists, applied mathematicians, computer scientists, sociologists,
and others. This interdisciplinarity arises both from the variety of
methods that can be applied, and the breadth of potential applications,
often requiring domain-specific knowledge
\autocite{porter_communities_2009}. Within this new field, the concept
of \emph{community} has been somewhat more formalized from the idea
above as a group of nodes (a \emph{subgraph}) with a high concentration
of edges connecting vertices within the group, and a low concentration
of edges with nodes outside the group
\autocite{fortunato_community_2010}.

The earliest analyses of communities were made by social scientists in
the early- to mid-twentieth century---for example Weiss and Jacobson's
analysis of the organizational structure of a government agency
\autocite{weiss_method_1955}. More developments were made by computer
scientists, who began developing graph partitioning algorithms in the
early 1970s to apply to problems in parallel computing and circuit
layout. In 2002, a seminal paper by Girvan and Newman
\autocite{girvan_community_2002} marked the entrance of the physics
community, and ushered in the modern age of community detection
\autocite{lancichinetti_community_2009}. Since then, the field has seen
rapid growth and the development of many new methods.

\section{Community detection methods}\label{community-detection-methods}

What follows is an overview of some of the many community detection
methods currently in use. The overview follows the taxonomy laid out in
a recent paper by Schaub et al. \autocite{schaub_many_2017}. The authors
identify four different perspectives on the problem of community
detection: (i) the \emph{cut-based perspective}, (ii) the \emph{(data)
clustering perspective}, (iii) the \emph{stochastic equivalence
perspective}, and (iv) the \emph{dynamical perspective}. The different
perspectives represent different approaches to the problem, often with
different kinds of data, different methods, and different goals. They
also represent to some degree the different research communities that
have been working on the problem.

\TODO{Add a note about how this might not be the only way to classify the problem space, but it is one way. Link to discussion in the evaluation section on the need for splitting the problem up.}

\subsection{The cut-based perspective}\label{the-cut-based-perspective}

Some of the earliest work in community detection was in the area of
circuit layout and design. A circuit can be represented as a graph
describing the signal flows between its components. The efficient layout
of of a circuit depends on partitioning the circuit into a fixed number
of similarly sized groups with a small number of edges between
groups---these inter-group edges are known as the \emph{cut}. Similar
problems can be found in load scheduling and parallel computing, where
tasks must be divided into different portions with minimal dependencies
between them. These need for these methods led to the development of the
Kernighan-Lin algorithm in 1970 \autocite{kernighan_efficient_1970},
which has become a classical method that is still frequently used. It
starts with an initial partition and attempts to optimize a quality
function \(Q\) representing the difference between intra-cluster edges
and inter-cluster edges, by swapping equal-sized subsets of vertices
between groups. The method works best if it starts with a decent initial
partition, so in modern use it is often used to refine partitions
obtained using other methods \autocite{fortunato_community_2010}.

The cut-based perspective has also seen the development of spectral
methods for graph partitioning. The spectrum (eigenvalues) of a graph's
adjacency matrix tend to be related to the connectivity of the graph,
and the associated eigenvectors can be used for both cut-based
partitioning and clustering. These methods typically make use of the
Laplacian matrix \(L\) of a connected graph \(G\): \(L = D - A\) where
\(A\) is the adjacency matrix of \(G\) and \(D\) is the diagonal degree
matrix with \(D_{ii} = \sum_{j}{A_{ij}}\). The \emph{Fiedler vector} is
the second-smallest eigenvector associated with the Laplacian matrix
\(L\); the spectral bisection method uses this vector to quickly
partition a graph into two groups with a low cut size
\autocite{fiedler_algebraic_1973}.

\subsection{The clustering
perspective}\label{the-clustering-perspective}

\TODO{conductance: originally developed with the cut-based perspective, but adapted as a local measure to be a clustering quality measure}

\TODO{hierarchical clustering? Schaub et al. don't mention it.}

\TODO{modularity}

\subsection{The stochastic equivalence
perspective}\label{the-stochastic-equivalence-perspective}

\subsection{The dynamical perspective}\label{the-dynamical-perspective}
