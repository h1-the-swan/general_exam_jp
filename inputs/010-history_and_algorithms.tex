\hypertarget{history}{\section{History of community
detection}\label{history}}

Across many disciplines of science, it is common to encounter data that
can best be represented as a network, with entities linked to each other
in some meaningful way, such as through association or flow. These
entities are represented by \emph{nodes} or \emph{vertices} connected to
each other with \emph{links} or \emph{edges}. This overall
representation is known as a \emph{network} or \emph{graph}.\footnote{The
  term \emph{graph} generally refers to a mathematical representation of
  data, while \emph{network} usually has additional connotations related
  to the meaning and context associated with the data
  \autocite{porter_communities_2009}; however, as is the case with many
  terms in this area, the distinction is not always made and the two are
  often used interchangeably.} The idea of community detection as a
research topic comes from a basic intuition that there exist in these
networks groups of nodes that are structurally more related to each
other than they are to members of other groups.

The field of \emph{network science} has emerged recently to study this
and related topics. This field is highly interdisciplinary, comprising
physicists, applied mathematicians, computer scientists, sociologists,
and others. This interdisciplinarity arises both from the variety of
methods that can be applied, and the breadth of potential applications,
often requiring domain-specific knowledge
\autocite{porter_communities_2009}. Within this new field, the concept
of \emph{community} has been somewhat more formalized from the idea
above as a group of nodes (a \emph{subgraph}) with a high concentration
of edges connecting vertices within the group, and a low concentration
of edges with nodes outside the group
\autocite{fortunato_community_2010}.

The earliest analyses of communities were made by social scientists in
the early- to mid-twentieth century---for example Weiss and Jacobson's
analysis of the organizational structure of a government agency
\autocite{weiss_method_1955}. More developments were made by computer
scientists, who began developing graph partitioning algorithms in the
early 1970s to apply to problems in parallel computing and circuit
layout. In 2002, a seminal paper by Girvan and Newman
\autocite{girvan_community_2002} marked the entrance of the physics
community, and ushered in the modern age of community detection
\autocite{lancichinetti_community_2009}. The Girvan and Newman algorithm
introduced in their paper involved successively calculating the
\emph{edge betweenness}---the number of shortest path between all nodes
that run along the edge---of all edges, then removing the edge with the
highest betweenness and repeating. The idea is that the edges with the
highest betweenness centralities are the ones that connect communities,
and the communities can be separated by this divisive algorithm. This
work inspired the development of modularity as a quality measure (see
section on \protect\hyperlink{the-clustering-perspective}{the clustering
perspective} below). Since then, the field has seen rapid growth and the
development of many new methods.

\section{Community detection methods}\label{community-detection-methods}

What follows is an overview of some of the many community detection
methods currently in use. The overview follows the taxonomy laid out in
a recent paper by Schaub et al. \autocite{schaub_many_2017}. The authors
identify four different perspectives on the problem of community
detection: (i) \protect\hyperlink{the-cut-based-perspective}{the
\emph{cut-based perspective}}, (ii)
\protect\hyperlink{the-clustering-perspective}{the \emph{(data)
clustering perspective}}, (iii)
\protect\hyperlink{the-stochastic-equivalence-perspective}{the
\emph{stochastic equivalence perspective}}, and (iv)
\protect\hyperlink{the-dynamical-perspective}{the \emph{dynamical
perspective}}. The different perspectives represent different approaches
to the problem, often with different kinds of data, different methods,
and different goals. They also represent, to some degree, the different
research communities that have been working on the problem.

\TODO{Add a note about how this might not be the only way to classify the problem space, but it is one way. It does not divide the methods cleanly, but neither do other classification. This is maybe because there are so many connections between methods. Link to discussion in the evaluation section on the need for splitting the problem up.}

\hypertarget{the-cut-based-perspective}{\subsection{The cut-based
perspective}\label{the-cut-based-perspective}}

Some of the earliest work in community detection was in the area of
circuit layout and design. A circuit can be represented as a graph
describing the signal flows between its components. The efficient layout
of of a circuit depends on partitioning the circuit into a fixed number
of similarly sized groups with a small number of edges between
groups---these inter-group edges are known as the \emph{cut}. Similar
problems can be found in load scheduling and parallel computing, where
tasks must be divided into different portions with minimal dependencies
between them. These need for these methods led to the development of the
Kernighan-Lin algorithm in 1970 \autocite{kernighan_efficient_1970},
which has become a classical method that is still frequently used. It
starts with an initial partition and attempts to optimize a quality
function \(Q\) representing the difference between intra-cluster edges
and inter-cluster edges, by swapping equal-sized subsets of vertices
between groups. The method works best if it starts with a decent initial
partition, so in modern use it is often used to refine partitions
obtained using other methods \autocite{fortunato_community_2010}.

The cut-based perspective has also seen the development of spectral
methods for graph partitioning. The spectrum (eigenvalues) of a graph's
adjacency matrix tend to be related to the connectivity of the graph,
and the associated eigenvectors can be used for both cut-based
partitioning and clustering. These methods typically make use of the
Laplacian matrix \(L\) of a connected graph \(G\): \(L = D - A\) where
\(A\) is the adjacency matrix of \(G\) and \(D\) is the diagonal degree
matrix with \(D_{ii} = \sum_{j}{A_{ij}}\). The \emph{Fiedler vector} is
the second-smallest eigenvector associated with the Laplacian matrix
\(L\); the spectral bisection method uses this vector to quickly
partition a graph into two groups with a low cut size
\autocite{fiedler_algebraic_1973}.

A cut-based measure for the quality of a partition is the
\emph{conductance}. The conductance of a subgraph \(S \in V\) for a
graph \(G(V, E)\) is:
\[\phi(S) = \frac{c(S, V \setminus S)}{\min(k_S, k_{V \setminus S})}\]
where \(c(S, V \setminus S)\) is the cut size of \(S\), and \(k_S\) and
\(k_{V \setminus S}\) are the total degrees of \(S\) and the rest of the
graph, respectively \autocite{schaeffer_graph_2007}. While this measure
was originally used globally to optimize a bisection of a graph, it has
also seen use as a local quality function to find good clusters around
certain nodes; in this way it can also be viewed as part of the
clustering perspective \autocite{schaub_many_2017}. Conductance has its
roots in computer science, and its use in network science appears still
to be especially popular in the computer science community
\autocites{schaeffer_graph_2007}{yang_defining_2015}.

\hypertarget{the-clustering-perspective}{\subsection{The clustering
perspective}\label{the-clustering-perspective}}

The clustering perspective comes from the world of data clustering, in
which data points are thought of as having ``distance'' between each
other based on their (dis)similarity, and the goal is to group together
data point that are close to each other. For community detection, this
distance is in relation to the connections between nodes in the network.
This perspective is related to but different from the cut-based
perspective above, which seeks to place divisions among the nodes so as
to form balanced groups with weak connections between groups.

A classical method with this perspective is \emph{hierarchical
clustering}, which when used on graphs yields a hierarchical
partitioning that can be viewed as a dendrogram. The common method uses
an agglomerative approach in which each node starts in its own cluster,
and they are joined together one by one based on some similarity measure
calculated using the graph's adjacency matrix. This approach to
community detection has several weaknesses. It necessarily infers a
hierarchical community structure even if one does not exist; the
hierarchy is not always easy to interpret; it often misclassifies nodes,
especially nodes with only one neighbor, which it tends to put it in its
own cluster; and it does not scale well to large networks
\autocite{fortunato_community_2010}.

The \emph{modularity} is a quality function originally developed as
stopping criterion for the Girvan-Newman algorithm (see
\protect\hyperlink{history}{section on history} above)
\autocites{newman_finding_2004}{newman_modularity_2006}. The algorithm
partitions a graph by successively removing links, but the original
formulation did not have a clear stopping point at which the communities
have been identified. The modularity compares an overall partitioning
against a null model. For a given graph \(G(V, E)\) and partitioning
\(\mathcal{C}\), the modularity is:
\[Q = \frac{1}{2m} \sum_{i,j \in V} \left(A_{ij} - \frac{k_i k_j}{2m}\right) \delta(C_i, C_j)\]
where \(m\) is the total number of edges in \(G\), \(A\) is the
adjacency matrix of \(G\), \(k_i\) is the degree of vertex \(i\), and
\(\delta(C_i, C_j)\) is a function that yields one if vertices \(i\) and
\(j\) are in the same community in \(\mathcal{C}\), zero otherwise. The
term \(k_i k_j / 2m\) represents the standard null model used for
modularity---the configuration model which preserves the degree sequence
of the original graph.

\hypertarget{the-stochastic-equivalence-perspective}{\subsection{The
stochastic equivalence
perspective}\label{the-stochastic-equivalence-perspective}}

\hypertarget{the-dynamical-perspective}{\subsection{The dynamical
perspective}\label{the-dynamical-perspective}}

\subsection{Future/open problems}\label{futureopen-problems}

\TODO{hierarchical. overlapping.}
