\section{History of community
detection}\label{history-of-community-detection}

Across many disciplines of science, it is common to encounter data that
can best be represented as a network, with entities linked to each other
in some meaningful way, such as through association or flow. These
entities are represented by \emph{nodes} or \emph{vertices} connected to
each other with \emph{links} or \emph{edges}. This overall
representation is known as a \emph{network} or \emph{graph}.\footnote{The
  term \emph{graph} generally refers to a mathematical representation of
  data, while \emph{network} usually has additional connotations related
  to the meaning and context associated with the data
  \autocite{porter_communities_2009}; however, as is the case with many
  terms in this area, the distinction is not always made and the two are
  often used interchangeably.} The idea of community detection as a
research topic comes from a basic intuition that there exist in these
networks groups of nodes that are structurally more related to each
other than they are to members of other groups.

The field of \emph{network science} has emerged recently to study this
and related topics. This field is highly interdisciplinary, comprising
physicists, applied mathematicians, computer scientists, sociologists,
and others. This interdisciplinarity arises both from the variety of
methods that can be applied, and the breadth of potential applications,
often requiring domain-specific knowledge
\autocite{porter_communities_2009}. Within this new field, the concept
of \emph{community} has been somewhat more formalized from the idea
above as a group of nodes (a \emph{subgraph}) with a high concentration
of edges connecting vertices within the group, and a low concentration
of edges with nodes outside the group
\autocite{fortunato_community_2010}.

The earliest analyses of communities were made by social scientists in
the early- to mid-twentieth century; for example Weiss and Jacobson's
analysis of the organizational structure of a government agency
\autocite{weiss_method_1955}. More developments were made by computer
scientists, who began developing graph partitioning algorithms in the
early 1970s to apply to problems in parallel computing and circuit
layout. In 2002, a seminal paper by Girvan and Newman
\autocite{girvan_community_2002} marked the entrance of the physics
community, and ushered in the modern age of community detection
\autocite{lancichinetti_community_2009}. Since then, the field has seen
rapid growth and the development of many new methods.

\TODO{Probably follow schaub et al to go through the different methods}
