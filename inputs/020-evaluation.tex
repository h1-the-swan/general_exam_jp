\section{Evaluation of community detection
methods}\label{evaluation-of-community-detection-methods}

The lack of consensus on exactly what a community is and what is meant
to be achieved by its detection has presented problems for the
evaluation of community detection methods. Still, attempts have been
made to systematically evaluate the performance and output of different
methods.

Evaluating the results of any community detection method can be thought
in terms of either \emph{internal} or \emph{external} validity. Measures
of \emph{internal} validity evaluate the output of a clustering
algorithm according to some quality measure that uses only the
properties of this output; these include modularity
\autocite{newman_finding_2004}, minimum description length
\autocite{rosvall_map_2010}, and conductance
\autocite{leskovec_empirical_2010}. These measures are designed to
measure how well the communities identified by a method adhere to some
mathematical definition of what a proper community structure should look
like. The problem with using these measures to evaluate and compare
methods is that these measures often serve as the objective functions
for the very algorithms we want to evaluate. While it may be useful to
use these quality measures to compare algorithms that are trying to
optimize the same function, it may not be fair to compare more broadly
than this. As there is no strict mathematical definition for a
community, different algorithms use different quality functions to
surface community structure, and those algorithms that optimize for
whatever measure we are using to evaluate would have an unfair
advantage. Some of these quality measures are discussed in the previous
section on existing community detection algorithms.

Because of this, much of the work around evaluating community detection
methods has focused on \emph{external} validity measures, in which the
input is a network with a known community structure. The evaluation in
this case measures how well the community detection method matches this
ground truth. These ``gold standard'' networks used for evaluation are
either (1) synthetic benchmark networks created with planted
communities, or (2) real-world networks with known metadata which are
treated as ground-truth communities. In either case, the results of a
community detection algorithm can be evaluated against the expected
structure using some comparison measure.

\subsection{Comparison measures}\label{comparison-measures}

Evaluating a community detection method against either a synthetic
benchmark network or a real-world network with known community structure
requires some measure of comparison between the clustering found by the
method and the ground truth clustering. The popular measures that have
been adopted fall into one of three categories: (1) measures based on
\emph{pair counting}, (2) measures based on \emph{set matching}, or (3)
measures based on \emph{information theory}
\autocites{meila_comparing_2007}{vinh_information_2010}. These are all
general measures comparing data (not just network) clusterings; they
work by viewing the network as data points with communities as cluster
assignments.

\emph{Pair counting measures} work by taking every possible pair of
nodes in the network and classifying them based on their co-occurrence
in the clusterings. Each of these categories is then counted:

\begin{itemize}
\tightlist
\item
  \(N_{11}\): the number of pairs that co-occur in the same cluster in
  both clusterings
\item
  \(N_{00}\): the number of pairs that do not co-occur in either
  clustering
\item
  \(N_{10}\) or \(N_{01}\): the number of pairs that co-occur in one
  clustering but not the other.
\end{itemize}

Examples of measures that use these counts include the Fowlkes-Mallows
index \autocite{fowlkes_method_1983} and the Rand index
\autocite{rand_objective_1971}. The Rand index, for example, is the
ratio of pairs correctly classified in both clusterings to the total
number of pairs:

\[\frac{N_{11} + N_{00}}{N_{11} + N_{00} + N_{10} + N_{01}}\]

This measure has a value between zero and one, with one representing
perfect agreement between the clusterings and zero representing no
agreement whatsoever. In practice, it is rare to see values on the lower
end of this range, so a transformation is usually applied that sets a
baseline that accounts for chance---this is known as the adjusted Rand
index.

\emph{Set matching measures} compare clusterings by finding matches
between the clusters---for example, by treating the cluster assignments
as labels and calculating the classification error rate. This approach
has problems when the two clusterings to be compared have different
numbers of clusters, however. Even within the clusters that match, these
measures only consider the matched part of each cluster pair, leaving
out the parts that do not match. For these reasons, these measures are
not very widely used
\autocites{meila_comparing_2007}{vinh_information_2010}.

\emph{Information theoretic measures} use elements of information theory
to compare clusterings.

\subsection{Synthetic benchmark
networks}\label{synthetic-benchmark-networks}

\subsection{Evaluation on real-world
networks}\label{evaluation-on-real-world-networks}

\subsection{Visualization}\label{visualization}
